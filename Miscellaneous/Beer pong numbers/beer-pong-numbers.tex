\documentclass{article}
\usepackage[utf8]{inputenc}
\usepackage{parskip}
\usepackage{lmodern}
\usepackage[T1]{fontenc}
\usepackage[english]{babel}
\usepackage{amsmath}
\usepackage{amsthm}
\usepackage{amssymb}
\usepackage{graphicx}

\title{History of Mathematics\\exercise set II}
\author{Elli Kiiski}

\begin{document}


{\large
Elli Kiiski
\par
Edited version of an exercise from the course History of Mathematics\\University of Helsinki
}
\vspace{0.5cm}


\section*{Beer pong numbers}

Beer pong numbers (also known as triangular numbers) are figurative numbers that can be represented by arranging red cups (or evenly spaced dots) into a shape of an equilateral triangle.

The series of beer pong numbers starts with $1,3,6,10,15,21...$ and the $n$th pong number can be constructed followingly
\begin{equation*}
\label{kolmiosumma}
    T_n = 1+2+3+...+n = \sum_{k=1}^n k = \frac{n(n+1)}{2}.
\end{equation*}
Let's show that the sum of any two consecutive beer pong numbers is always a square number. We have
\begin{align*}
    T_n + T_{n+1} & = \frac{n(n+1)}{2} + \frac{(n+1)(n+2)}{2}\\
    & = \frac{1}{2}(n^2+n+n^2+3n+2)\\
    & = n^2+2n+1\\
    & = (n+1)^2,
\end{align*}
and as we see, the sum indeed makes a square of $(n+1)$. Also,  we notice that
\begin{align*}
    T_{n+1} - T_n & = \sum_{k=1}^{n+1} k - \sum_{k=1}^n k = n+1,
\end{align*}
which means that the sum of two consecutive beer pong numbers equals the square of their difference. Formally
\begin{equation*}
    T_n + T_{n+1} = (T_{n+1} - T_n)^2.
\end{equation*}
Figure \ref{fig:pong} shows an example with $n=2$, commonly known as overtime layout ($T_2=3$) plus 1vs1 layout ($T_3=6$).
\begin{figure}
    \centering
    \includegraphics[width= 120mm]{pong-numbers.eps}
    \caption{Sum of two consecutive beer pong numbers illustrated.}
    \label{fig:pong}
\end{figure}

\end{document}
