\documentclass{article}
\usepackage[utf8]{inputenc}
\usepackage{parskip}
\usepackage{lmodern}
\usepackage[T1]{fontenc}
\usepackage[english]{babel}
\usepackage{amsmath}
\usepackage{amsthm}
\usepackage{amssymb}
\usepackage{graphicx}

\begin{document}

{\large
Elli Kiiski
\par
An exercise from the course History of Mathematics\\University of Helsinki
}
\vspace{0.5cm}

\section*{Fibonacci sequence and golden ratio}

The famous Fibonacci sequence starts with 1, 1, 2, 3, 5, 8, 13... and its members have the relationship $F_{n+1}-F_n=F_{n-1}$. Based on this, let's prove that
\begin{equation*}
    \lim_{n\rightarrow\infty}\frac{F_{n+1}}{F_n}=\Phi\,\text{, where $\Phi$ is the golden ratio.}
\end{equation*}

By the relationship above
\begin{equation*}
    F_{n+1}-F_n=F_{n-1} \quad \Leftrightarrow \quad \frac{F_{n+1}}{F_n}-1=\frac{F_{n-1}}{F_n} \quad \Leftrightarrow \quad \frac{F_{n+1}}{F_n}=1+\frac{F_{n-1}}{F_n}
\end{equation*}
and using this result recursively, we get
\begin{align*}
    \frac{F_{n+1}}{F_n} & = 1+\frac{F_{n-1}}{F_n} = 1+\frac{1}{\frac{F_n}{F_{n-1}}} = 1+\frac{1}{1+\frac{F_{n-1}}{F_{n-2}}}\\
    & = 1+\frac{1}{1+\frac{1}{\frac{F_{n-2}}{F_{n-1}}}} = 1+\frac{1}{1+\frac{1}{1+\frac{F_{n-2}}{F_{n-3}}}} = 1+\frac{1}{1+\frac{1}{1+\dots}}\,,
\end{align*}

which is a known form for golden ratio $\Phi$.

\end{document}