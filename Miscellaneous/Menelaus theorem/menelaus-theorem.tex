\documentclass{article}
\usepackage[utf8]{inputenc}
\usepackage{parskip}
\usepackage{lmodern}
\usepackage[T1]{fontenc}
\usepackage[english]{babel}
\usepackage{amsmath}
\usepackage{amsthm}
\usepackage{amssymb}
\usepackage{graphicx}

\begin{document}

{\large
Elli Kiiski
\par
An exercise from the course History of Mathematics\\University of Helsinki
}
\vspace{0.5cm}

\section*{Menelaus's theorem}

\begin{figure}[!htb]
    \centering
    \includegraphics[width =60mm]{menelaus.eps}
    \caption{The setting for Menelaus's theorem.}
    \label{fig:menelaus}
\end{figure}

Menelaus's theorem states that for a line intersecting triangle $ABC$ in points $D$, $E$ and $F$ (see figure \ref{fig:menelaus}) holds
\begin{equation*}
    \frac{AD}{BD}\cdot\frac{BE}{CE}\cdot\frac{CF}{AF}=1\,.
\end{equation*}

Let's prove this by drawing a line $BP$ parallel to line $AC$ and hence forming two sets of similar triangles: $ADF \sim BDP$ and $BPE \sim FEC$.

By the similarity we have $\frac{AF}{BP}=\frac{AD}{BD}$ and $\frac{BP}{CF}=\frac{BE}{CE}$. Multiplying these two equations with each other we get
\begin{equation*}
    \frac{AF}{BP}\cdot\frac{BP}{CF}=\frac{AD}{BD}\cdot\frac{BE}{CE} \quad \Leftrightarrow \quad \frac{AF}{CF}=\frac{AD}{BD}\cdot\frac{BE}{CE} \quad \Leftrightarrow \quad 1=\frac{AD}{BD}\cdot\frac{BE}{CE}\cdot\frac{CF}{AF}\,,
\end{equation*}
proving the Menelaus's theorem.

\end{document}